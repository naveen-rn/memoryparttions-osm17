\section{Introduction}
\label{src:intro}

Emerging systems support multiple kinds of memory with different
performance and capacity characteristics. For example, the latest
Many Integrated Core (MIC) processor, the Intel Xeon Phi,
code named Knights Landing (KNL)~\cite{KNL}, combines the traditional
off-package DDR memory with the increased high bandwidth on-package
memory called as the Multi-Channel DRAM~\cite{MCDRAM} (MCDRAM). MCDRAM
offers high bandwidth (up to 4X more than DDR4) with limited
capacity (upto 16GB). Vendors provide their own programming approaches
using external libraries like CUDA~\cite{cuda} and Memkind~\cite{memkind}
to differentiate the different kinds of memory or users are forced to
use low level programming approaches. Irrespective of the selection these
approaches would turn applications non-portable. In response, it is for
the programming library interfaces like OpenSHMEM~\cite{osm} to provide
a more consistent and portable interface for accessing different kinds
of memory on tiered memory systems.

In this paper, we define OpenSHMEM runtime changes and extensions to
support different kinds of memory for the symmetric heap. The OpenSHMEM
memory model creates the symmetric heap during program execution on a
memory region determined by the implementation. The runtime changes
described in this paper would allow users to determine a region on
which the symmetric heap can be created. We call these user-determined
region as \emph{Symmetric Memory Partitions}. Each memory partitions
feature a list of traits to define their characteristics and memory
kind being one of those featured traits.

The major contributions of this work are:
\begin{itemize}
    \item proposing a set of new runtime changes and extensions in
    OpenSHMEM to support different kinds of memory on tiered systems;
    \item implementation of the proposed feature in
    Cray SHMEM~\cite{csma}, a vendor-specific OpenSHMEM implementation
    from Cray Inc;
    \item performance regression analysis on creating multiple symmetric
    partitions; and
    \item performance analysis of these proposed extensions using
    modified Parallel Research Kernels (ParRes)~\cite{parres} on Intel
    KNL processors as an example use-case for emerging systems.
\end{itemize}

This paper is organized as follows.
In Section~\ref{src:bground} we give a brief overview on the current
OpenSHMEM memory model and in Section~\ref{src:smempart} we propose the
new runtime changes and extensions in OpenSHMEM to support different kinds
of memory for future emerging systems. In Section~\ref{src:knl} we provide
a brief overview of Intel KNL architecture with different modes of memory
configuration. In Section~\ref{src:drelated}, we motivate the need for
the symmetric memory partitions in OpenSHMEM by assessing the performance
benefits on using the different kinds of memory in Intel KNL processors
with different mode of configurations. In Section~\ref{src:implement}, we
follow up with the implementation of the proposed changes in Cray SHMEM.
In the experimental results presented in Section~\ref{src:perf} we use
the modified ParRes OpenSHMEM kernels for testing the performance of the
proposed routines along with the performance regression analysis with the
introduction of multiple symmetric heaps. We discuss related work
in Section~\ref{src:relate} and we conclude in Section~\ref{src:conclusion}.
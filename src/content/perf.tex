\section{Performance Regression Analysis}
\label{src:regression}
Creating multiple symmetric memory partitions would result in
multiple symmetric heaps on each PE. The symmetric heap maintenance
includes memory registration with NIC and offset maintenance on
each PE. Though upcoming communication layers like
libfabrics~\cite{libfabrics} have support for Scalable memory
Registration, Cray SHMEM implementation over DMAPP uses basic memory
registration and hence there is a increase in memory footprint on
DMAPP-level on maintaining symmetric heaps on each PE.

\begin{algorithm}[!h]
\begin{algorithmic}
    \Procedure{\texttt{shmem\_putmem}}
    {void *dest, const void *src, size\_t nblks, int pe\_id}\;
        dmapp\_seg\_desc\_t *target\_seg = \texttt{NULL}\;
        dmapp\_type\_t type = \texttt{DMAPP\_BYTE}\;
        \If {\normalfont (dest.segment $\equiv$
                         \texttt{DATA\_SEGMENT})} {
            target\_seg = \texttt{DATA\_SEGMENT}\;
        } \ElseIf {\normalfont (dest.segment $\equiv$
                         \texttt{SHEAP\_SEGMENT})} {
            target\_seg = \texttt{SHEAP\_SEGMENT}\;
        } \Else {
            segment\_identified = \textit{false}\;
            \For {\normalfont (int i = 1; i $\leq$ N; i++)}{
                \If{\normalfont (dest.segment $\equiv$
                    \texttt{USE\_SEGMENT\_i})} {
                    target\_seg = \texttt{USE\_SEGMENT\_i})\;
                    segment\_identified = \textit{true}\;
                }
            }
            \If{\normalfont(segment\_identified $\equiv$
                    \textit{false})} {
                abort()\;
            }
        }
        dmapp\_put(dest, target\_seg, pe\_id, src, nelems, type)\;
        return\;
    \EndProcedure
    \caption{Lookup logic with N symmetric memory partitions per PE}
    \label{algo:normal-lookup}
\end{algorithmic}
\end{algorithm}

Apart from the increase in memory footprint on memory registration,
identifing the correct symmetric heap segment before performing an
underlying communication operation with symmetric heap lookup on each
RMA and AMO operation is expected to introduce performance
regression. Algorithm~\ref{algo:normal-lookup} shows the introduction
of lookup trying to match the correct symmetric heap segment on the
\texttt{shmem\_putmem} operation. Similar, lookups are introduced in
all OpenSHMEM RMA and AMO operations. \texttt{USER\_SEGMENT} refers to
the symmetric heap segments created on user-defined memory partitions.

We measured the performance regression behind this additional lookup
operation using a modified OSU microbenchmark~\cite{pgas-benchmarks}
on a Cray XC system with Intel KNL processors using 2 PEs with 1 PE per
node.
The normal OSU microbenchmark selects the memory segment (either
\texttt{DATA\_SEGMENT} or \texttt{SHEAP\_SEGMENT}) of the buffer
per job. We modified this \texttt{shmem\_put} benchmark by creating
one destination buffer per \texttt{N} partitions and we randomly select
the destination buffer from different partitions for every iteration.
We timed the average performance of all the iterations of the
\texttt{shmem\_putmem} operation.

Figure~\ref{graph:lookup} shows the
performance difference between using 1 and 7 partitions on very small
data sies. There are average performance variations is around 2\% to
3\% and this variation can be attributed as noise. But, if we increase
the number of partitions to 127, we could see the variations as high as
6\%. We expect users to create only one partition per memory kind
and not \texttt{N} unnecessary partitions. Moreover, as mentioned in
Table~\ref{tab:const} the \texttt{SHMEMX\_MAX\_PARTITIONS} library constant
in Cray SHMEM is 7.

\begin{figure*}[t!]
    \centering
    \includegraphics[width=\linewidth]{graph/osu-put.eps}
    \caption{Performance difference on using destination buffers
    from 1 partition and 7 partitions
    on modified OSU Put microbenchmark}
    \label{graph:lookup}
    \vspace{-20pt}
\end{figure*}

We observe similar performance variation on small data sizes on other
RMA and AMO operations. For larger message sizes, the segment lookup
does not contribute much for any performance variation.

\section{Performance Analysis}
\label{src:perf}
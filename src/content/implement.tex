\section{Symmetric Memory Partitions in Cray SHMEM}
\label{src:implement}
\begin{table}[h!]
\centering
\begin{tabular}{|c c|}\hline
    Library Constants & values \\[0.5ex]
    \hline\hline
    SHMEMX\_MAX\_PARTITIONS & 7 \\
    SHMEMX\_MAX\_PARTITION\_ID & 127 \\
    \hline
\end{tabular}
\caption{Memory partition specific OpenSHMEM library constants}
\label{tab:const}
\end{table}

The symmetric memory partitions features introduced in Section
~\ref{src:smempart} are available as \texttt{SHMEMX} prefixed
features in Cray SHMEM. Table~\ref{tab:const} provides information
on the library constants used to determine the range of partition
IDs that can be used and the maximum number of partitions that can
be created per job.

\begin{table}[h!]
\centering
\begin{tabular}{|l l l|}\hline
    Traits & values & Explanation\\[0.5ex]
    \hline\hline
    \multirow{3}{4em}{KIND} & NORMALMEM  & primary memory kind for
                                           the node \\
                            & FASTMEM    & faster memory in addition
                                           to NORMALMEM \\
                            & SYSDEFAULT & system defined memory \\
                            &&\\
    \multirow{4}{4em}{POLICY} & MANDATORY   & abort if requested memory
                                              kind not available \\
                              & PREFERRED   & use other memory kinds if
                                              requested kind fails \\
                              & INTERLEAVED & page allocation interleaved
                                              across NUMA domains \\
                              & SYSDEFAULT  & system defined policy \\
    \hline
\end{tabular}
\caption{Available partition traits in Cray SHMEM}
\label{tab:trait}
\end{table}

Table~\ref{tab:trait} refers the values for the user-determined traits
in defining the characteristics of the symmetric partition. SIZE and PGSIZE
can be any appropriate symmetric heap size and any available page size in
the system represented as bytes. KIND determines the type of memory to be
used and POLICY the memory kind allocation policy.

On Intel KNL based system, NORMALMEM refers to DDR4 and FASTMEM refers to
MCDRAM. If MCDRAM is requested and the requested memory kind if not available,
based on the allocation policy Cray SHMEM either aborts the job or look for
other alternate memory kind to be used. As mentioned in
section~\ref{src:knl/cluster}, number of NUMA domains depends on the NUMA
clustering either through quadrant or SNC mode. INTERLEAVED is used to shift
allocation across different NUMA domains. On quadrant mode, since MCDRAM is
available as a single NUMA domain, INTERLEAVED will allocate on a single
NUMA domain. On SNC2 and SNC4 mode, allocation interleaves across 2 and 4
NUMA domains respectively.

All the available kinds of memory are identified and the environmental
variables are queried during \texttt{shmem\_init} operation. Memory kind
identification is performed using \texttt{numactl} system calls. SYSDEFAULT
refers to the system defined memory use based on \texttt{numactl} system
calls and not the implementation default values.
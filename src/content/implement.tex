\section{Symmetric Memory Partitions in Cray SHMEM}
\label{src:implement}
\begin{table}[h!]
\centering
\vspace{-20pt}
\begin{tabular}{|c c|}\hline
    Library Constants & Values \\[0.5ex]
    \hline\hline
    SHMEMX\_MAX\_PARTITIONS & 7 \\
    SHMEMX\_MAX\_PARTITION\_ID & 127 \\
    \hline
\end{tabular}
\caption{Memory partition specific OpenSHMEM library constants}
\label{tab:const}
\end{table}
\vspace{-20pt}

The symmetric memory partition features introduced in Section
~\ref{src:smempart} are available as \texttt{SHMEMX} prefixed
features in Cray SHMEM. Table~\ref{tab:const} provides information
on the library constants used in Cray SHMEM to determine the
range of partition ID and the maximum number of partitions per job.

\begin{table}[h!]
\centering
\vspace{-20pt}
\begin{tabular}{|l l l|}\hline
    Traits & Values & Explanation\\[0.5ex]
    \hline\hline
    \multirow{3}{4em}{KIND} & NORMALMEM  & primary memory kind for
                                           the node \\
                            & FASTMEM    & faster memory in addition
                                           to NORMALMEM \\
                            & SYSDEFAULT & system defined memory \\
                            &&\\
    \multirow{4}{4em}{POLICY} & MANDATORY   & abort if requested memory
                                              kind not available \\
                              & PREFERRED   & use other memory kinds if
                                              requested kind fails \\
                              & INTERLEAVED & page allocation interleaved
                                              across NUMA domains \\
                              & SYSDEFAULT  & system defined policy \\
    \hline
\end{tabular}
\caption{Available partition traits in Cray SHMEM}
\label{tab:trait}
\end{table}
\vspace{-20pt}

User-defined SIZE and PGSIZE traits are any appropriate symmetric heap size
and available page size in the system represented as bytes.
Table~\ref{tab:trait} refers the values for KIND and POLICY traits. KIND
determines the type of memory and POLICY the memory kind allocation policy.

On Intel KNL systems, NORMALMEM refers to DDR4 and FASTMEM refers to
MCDRAM. If MCDRAM is requested and the requested kind if unavailable,
based on the allocation policy Cray SHMEM either aborts or looks for
other possible alternatives. INTERLEAVED is used to shift allocation
across
different NUMA domains. As mentioned in section~\ref{src:knl/cluster},
number of NUMA domains depends on the NUMA clustering either through
quadrant or SNC configurations. On quadrant mode, since MCDRAM is
available as a
single NUMA domain, INTERLEAVED will allocate on a single NUMA domain.
On SNC2 and SNC4 mode, allocation interleaves across 2 and 4 NUMA domains
respectively.

All the available kinds of memory are identified and the environment
variables are queried during \texttt{shmem\_init} operation.
\texttt{numactl} controls the NUMA policy for processes or shared memory
with NUMA policy aware kernels. Memory kind identification in Cray SHMEM
is performed using \texttt{numactl}. There are no implementation defined
default values. SYSDEFAULT refers to the system defined memory use based
on \texttt{numactl} system calls.
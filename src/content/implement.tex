\section{Symmetric Memory Partitions in Cray SHMEM}
\label{src:implement}
\begin{table}[h!]
\centering
\vspace{-20pt}
\begin{tabular}{|c c|}\hline
    Library Constants & Values \\[0.5ex]
    \hline\hline
    \texttt{SHMEMX\_MAX\_PARTITIONS} & 7 \\
    \texttt{SHMEMX\_MAX\_PARTITION\_ID} & 127 \\
    \hline
\end{tabular}
\caption{Memory partition specific library constants in Cray SHMEM}
\label{tab:const}
\end{table}
\vspace{-20pt}

The symmetric memory partition features introduced in Section
~\ref{src:smempart} are available as \texttt{SHMEMX} prefixed
features in Cray SHMEM. Table~\ref{tab:const} provides information
on the library constants used in Cray SHMEM to determine the
range of partition ID and the maximum number of partitions per job.

\begin{table}[h!]
\centering
\vspace{-20pt}
\begin{tabular}{|l l l|}\hline
    Traits & Values & Explanation\\[0.5ex]
    \hline\hline
    \multirow{3}{4em}{\texttt{KIND}} & \texttt{NORMALMEM}  & primary memory kind for
                                                             the node \\
                            & \texttt{FASTMEM}    & faster memory in addition
                                                    to \texttt{NORMALMEM} \\
                            & \texttt{SYSDEFAULT} & system defined memory \\
                            &&\\
    \multirow{4}{4em}{\texttt{POLICY}} & \texttt{MANDATORY}
                                                     & abort if requested memory
                                                       kind not available \\
                              & \texttt{PREFERRED}   & use other memory kinds if
                                                       requested kind fails \\
                              & \texttt{INTERLEAVED} & page allocation interleaved
                                                       across NUMA domains \\
                              & \texttt{SYSDEFAULT}  & system defined policy \\
    \hline
\end{tabular}
\caption{Available partition traits in Cray SHMEM}
\label{tab:trait}
\end{table}
\vspace{-20pt}

User-defined \texttt{SIZE} and \texttt{PGSIZE} traits are any appropriate
symmetric heap size
and available page size in the system represented as bytes.
Table~\ref{tab:trait} refers the values for \texttt{KIND} and
\texttt{POLICY} traits. %\texttt{KIND}
%determines the type of memory and \texttt{POLICY} the memory kind
%allocation policy.

On Intel KNL systems, \texttt{NORMALMEM} refers to DDR4 and
\texttt{FASTMEM} refers to
MCDRAM. If the requested kind if unavailable,
based on the allocation policy Cray SHMEM either aborts or looks for
other possible alternatives. \texttt{INTERLEAVED} is used to shift
allocation
across
different NUMA domains. As mentioned in section~\ref{src:knl/cluster},
number of NUMA domains depends on the cluster configuration modes.
% either through
%quadrant or SNC configurations.
On quadrant mode, since MCDRAM is
available as a
single NUMA domain, \texttt{INTERLEAVED} will allocate on a single
NUMA domain.
On SNC2 and SNC4 mode, allocation interleaves across 2 and 4 NUMA
domains
respectively.

All the available kinds of memory are identified and the environment
variables are queried during \texttt{shmem\_init} operation. On NUMA
policy aware kernels,
\texttt{numactl} controls the NUMA policy for processes.% on shared memory
%with NUMA policy aware kernels.
Memory kind identification in Cray SHMEM
is performed using \texttt{numactl}. There are no implementation defined
default values. \texttt{SYSDEFAULT} refers to the system defined memory
use based
on \texttt{numactl} system calls.
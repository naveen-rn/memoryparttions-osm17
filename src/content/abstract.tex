To extract the best performance on emerging tiered memory systems,
it is essential for applications to use the different kinds of
memory available on the system. OpenSHMEM memory model consists of
data objects that are private to each \emph{Processing Element}
(PE) and data objects that are remotely accessible by all PEs. The
remotely accessible data objects are called \emph{Symmetric Data
Objects}. They are allocated on a memory region called as
\emph{Symmetric Heap} and it is created during program execution
on a region determined by the OpenSHMEM implementation. This paper
proposes a new feature called \emph{Symmetric Memory Partitions} to
enable users to determine the size along with other memory traits
for creating the symmetric heap, where the kind of memory is one of
the featured traits. Moreover, this paper uses Intel KNL as an
example use case for emerging tiered memory systems. This paper also
describes the implementation symmetric memory partitions in
Cray SHMEM and use different OpenSHMEM microbenchmark kernels to
show the benefits of selecting the memory region for the symmetric
heap.
%uses Intel KNL as
%an example use case for emerging systems.
%and describe the
%implementation of symmetric memory partitions in Cray SHMEM along with
%a list of traits that are used to define the characteristics of each
%partition and the kind of memory being one of the featured traits.
%This paper uses Intel KNL as

%This paper also uses different OpenSHMEM microbenchmark kernels to
%show the benefits of memory partitions in OpenSHMEM in utilizing the
%different performance and capacity characteristics on emerging systems
%with Intel KNL as an example use case.
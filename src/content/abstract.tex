OpenSHMEM memory model consists of data objects which are private
to each \emph{Processing Elements} (PEs) and data objects that are
remotely accessible by all PEs. The remotely accessible data objects
are called as \emph{Symmetric Data Objects} and are allocated on a
special memory region called \emph{Symmetric Heap}. The symmetric heap
is created during program execution by the OpenSHMEM implementation.
This paper proposes a new feature called \emph{Symmetric Memory
Partitions} to enable users to determine the size and other memory
traits for creating one or multiple symmetric heaps per PE. Moreover,
we describe the implementation of Symmetric Memory Partitions in
Cray SHMEM along with a list of traits that are used to define the
characteristics of each partition and the kind of memory being one of
the featured traits. This paper also uses different OpenSHMEM
microbenchmark kernels to show the benefits of memory partitions in
OpenSHMEM in utilizing the different performance and capacity
characteristics on emerging systems with Intel KNL as an example use
case.
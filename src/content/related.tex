\section{Related Work}
\label{src:relate}
In this paper we studied the reason behind having OpenSHMEM features to
utilize different kinds of memory. The need for these features for the
upcoming system architectures are not something specific to OpenSHMEM.
Other programming libraries started discussing on adding such similar
features in their specifications. For example, OpenMP~\cite{openmp}
has a memory model working group to discuss on the runtime changes
needed for supporting multiple kinds of memory for future architectures.
Similarly, Cray MPICH~\cite{cray-mpich} which is an optimized MPI
implementation for Cray systems have improvised the functionality of
\texttt{MPI\_Alloc\_mem} to enable users to select the kind of memory to
be used for allocating the requested memory size. ~\citeauthor{libfabrics}
provides detailed explanation on the performance of using different
memory kinds using \texttt{MPI\_Alloc\_mem} on two real world applications
WOMBAT and SNAP.
%need to fix citataions correctly

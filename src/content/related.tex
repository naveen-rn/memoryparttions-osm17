\section{Related Work}
\label{src:relate}
%In this paper we studied the necessity for basic OpenSHMEM changes
%to utilize different kinds of memory.
The need for utilizing different kinds of memory %these features
in upcoming system architectures is not specific to
OpenSHMEM. %Other programming libraries are planning on such similar
%features in their specifications.
%For example,
OpenMP~\cite{openmp} Affinity subcommittee proposes changes
for memory management~\cite{omp-tr5} support for future architectures.
%discuss on the runtime changes
%needed for supporting multiple kinds of memory for future architectures.
Similarly Cray MPICH, an optimized MPI
implementation for Cray systems have improvised the functionality of
\texttt{MPI\_Alloc\_mem} routine for allocating the requested
memory size on user-determined
memory kind. ~\citeauthor{cug17-krishna}
provides detailed explanation on the performance benefits of using
different
memory kinds using \texttt{MPI\_Alloc\_mem} on real world
applications WOMBAT and SNAP.

Similarly, the concept of creating multiple symmetric heaps in OpenSHMEM
is not unique to the proposal introduced in this paper.
~\citeauthor{osm-spaces} proposes \texttt{Teams} and
\texttt{Memory Spaces} in OpenSHMEM.
OpenSHMEM Teams are PE subsets and Memory Spaces are team-specific
symmetric heaps. Cray SHMEM already supports OpenSHMEM
Teams~\cite{knaak2015}
as \texttt{SHMEMX} prefixed features. With the introduction of
symmetric memory partitions, it
is logical to understand the possibilities of combining
memory partitions as part of OpenSHMEM Teams and introducing multiple
symmetric heaps on each partition in the form of Memory Spaces.
%need to fix citataions correctly

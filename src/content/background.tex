\section{Background}
\label{src:bground}

In this section, we provide a brief introduction to Cray SHMEM.% and an
%overview to the existing memory model in the OpenSHMEM standard
%specification.

\subsection{Cray SHMEM}
\label{src:bg/crayshmem}
OpenSHMEM is a Partitioned Global Address Space~\cite{pgas} (PGAS) library
interface specification which is a culmination of a unification effort
among various vendors and users in the SHMEM programming community. It
provides an Application Programming Interface (API) for the SHMEM libraries
to support one-sided point-to-point data communication with a chief aim of
performance and portability. %It follows an SPMD-like execution model.
%, where
%all the processes or Processing Elements(PEs) are launched at the begining
%of the program; each PE executes the same code and the number of PEs remains
%unchanged during the execution of the program. The main features of OpenSHMEM
%API includes the support for:
%\begin{itemize}
%    \item remote memory access through one-sided point-to-point blocking and
%    non-blocking communication for both continguous and strided data transfers;
%    \item atomic memory operations;
%    \item collective communication;
%    \item synchronization and memory ordering; and
%    \item mutual exclusion through distributed locking
%\end{itemize}

There are various optimized production-grade closed source as well as open source
OpenSHMEM implementations available. Cray SHMEM~\cite{csma} is a vendor-based
closed source OpenSHMEM implementation from Cray Inc., available as part of the
Cray Message Passing Toolkit~\cite{mpt} (MPT). It is OpenSHMEM
specification version-1.3~\cite{osm13} compliant and implemented over
DMAPP~\cite{dmapp}, an
optimized communication layer for Cray architectures. %Apart from the OpenSHMEM standard
%specific features, it provides support for the following features as \texttt{SHMEMX}
%prefixed extensions:
%\begin{itemize}
%    \item thread-safety in OpenSHMEM;
%    \item put with signal communication;
%    \item OpenSHMEM PE subsets called as Teams; and
%    \item Team-based collective communication extensions;
%\end{itemize}


\section{OpenSHMEM Memory Model}
\label{src:mmodel}
\begin{figure}[!h]
    \vspace{-30pt}
    \hspace*{5mm}
    \includegraphics[scale=0.20]{image/osm-mmodel.png}
    \vspace{-25pt}
    \caption{OpenSHMEM Memory Model}
    \vspace{-20pt}
    \label{fig:mmodel}
\end{figure}

As shown in Fig.~\ref{fig:mmodel}, an OpenSHMEM program consists of
two types of data objects: private and remotely
accessible data objects.
The private data objects are local
to a particular PE and are accessible by only that PE. It follows the
same memory model of the base programming language\footnote{With the
proposed deprecation of Fortran support in OpenSHMEM, in this paper
there is no reference to Fortran routines.} (C or C++). The
remotely accessible data objects are called
\emph{Symmetric Data Objects} and are accessible by all PEs.
%using the
%OpenSHMEM routines.
Each symmetric data object has a corresponding object on all PEs
with same %variable
name, size and data type.% on all PEs.
The following variables are considered
Symmetric Data Objects:
\begin{itemize}
    \item global or static variable on C/C++ and not defined in DSO; and
    \item data allocated by \texttt{shmem\_malloc} OpenSHMEM routines
\end{itemize}

The data allocated by \texttt{shmem\_malloc} collective OpenSHMEM
routines are placed on a special memory region called \emph{Symmetric
Heap}. There is one symmetric heap on every PE created during the program
execution on a memory region determined by the OpenSHMEM implmentation.
It may reside on different memory regions on different PEs. Users
control only the size of the symmetric heap using
\texttt{SMA\_SYMMETRIC\_SIZE} environment variable.
On systems with heterogeneous memory architectures, the symmetric heap
can be associated with different memory architectures on different PEs.
This can lead to highly variable communication costs for PEs accessing
objects on symmetric heaps.
%Except the
%size of the symmetric heap determined by \texttt{SMA\_SYMMETRIC\_SIZE}
%environment variable, users have no other control on any property of
%the symmetric heap.

%Fig.~\ref{fig:mmodel}, shows the different types of data objects in
%OpenSHMEM. Global and Static variables are allocated on the
%\emph{data segment}, while the variables with data allocated using
%\texttt{shmem\_malloc} routines are placed on the \emph{symmetric heap
%segment}. Variables on both the data and symmetric heap segments are
%remotely accessible. Fig.~\ref{fig:mmodel} also shows that there is only
%one symmetric heap per PE. %Local variables are allocated on the local
%data objects which retains the same memory model from the underlying
%base language.

\subsection{Need for Changes in OpenSHMEM Memory Model}
\label{src:mmodel/drelated}
As mentioned in Section~\ref{src:knl/config}, MCDRAM in Intel KNL can be
configured either as cache or as addressable memory. While configuring as
cache is a convenient way to port existing applications on to KNL based
systems, it is more suitable only for applications that are optimized for
cache utilizations and with small memory footprint.

The flat mode configuration is suitable for memory bandwidth bound
applications. Taking advantage of the high bandwidth offered by MCDRAM by
making it available as a distinct NUMA node and re-designing an application
can improve the performance. Based on the MCDRAM utilization, memory
bandwidth bound applications are of two types:
\begin{itemize}
    \item the entire memory of the application can fit in the MCDRAM; and
    \item capable of identifying specific bandwidth bound buffers and data
    access patterns,
    %applications capable of partitioning memory usage into
    %bandwidth critical and normal part,
    with the bandiwdth critical part
    allocated on MCDRAM
\end{itemize}
The current OpenSHMEM memory model does not handle both the above mentioned
categories for utilizing MCDRAM in flat mode configuration. And it is not
for OpenSHMEM to handle anything special for cache mode configuration.

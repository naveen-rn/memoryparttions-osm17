\section{OpenSHMEM Memory Model}
\label{src:mmodel}
\begin{figure}[!h]
    \vspace{-20pt}
    \hspace*{5mm}
    \includegraphics[scale=0.20]{image/osm-mmodel.png}
    \vspace{-25pt}
    \caption{OpenSHMEM Memory Model}
    \label{fig:mmodel}
\end{figure}

Based on accessibility, an OpenSHMEM program consists of two types of data
objects; remotely accessible and private data objects. The private data objects
are local to a particular PE and are accessible by only that PE.
These local data objects follow the same memory model of the base programming
language (C or Fortran). The remotely accessible data objects are accessible
by any PE using the OpenSHMEM extensions and they are called as \emph{Symmetric
Data Objects}. Any,
\begin{itemize}
    \item global or static variable on C/C++ and not defined in DSO; and
    \item data allocated by \texttt{shmem\_malloc} or \texttt{shpalloc}
    OpenSHMEM extensions on C/C++ and Fortran language respectively
\end{itemize}
is considered a symmetric data object. And each symmetric data object has a
corresponding object with same variable name, size and data type on all PEs

The data allocated by \texttt{shmem\_malloc} and \texttt{shpalloc} collective
OpenSHMEM extensions are placed on a special memory region called \emph{Symmetric
Heap}. There is one symmetric heap on every PE created during the program execution
on a memory region determined by the OpenSHMEM implmentation. It may reside on
different memory regions on different PEs. Except the size of the symmetric heap
determined by \texttt{SMA\_SYMMETRIC\_SIZE} environmental variable, users have no
other control on the symmetric heap.

Figure~\ref{fig:mmodel}, shows the different types of data objects available in
OpenSHMEM. Global and Static variables are allocated on the \emph{data segment},
while the the variable with data allocated using \texttt{shmem\_malloc} and
\texttt{shpalloc} extensions are placed on the \emph{symmetric heap segment}.
Variables on both the data and symmetric heap segments are remotely accessible.
Figure~\ref{fig:mmodel} also shows that there is only one symmetric heap segment
per PE. Local variables are allocated on the local data objects which retains the
same memory model from the underlying base language.

\subsection{Need for Heterogeneous Memory Support in OpenSHMEM}
\label{src:mmodel/drelated}
As mentioned in Section~\ref{src:knl/config}, MCDRAM in Intel KNL can be
configured either as cache or as addressable memory. While configuring as
cache is a convenient way to port existing applications on to KNL based
systems, it is more suitable only for applications that are optimized for
cache utilizations and with small memory footprint.

The flat mode configuration is suitable for memory bandwidth bound
applications. Taking advantage of the high bandwidth offered by MCDRAM by
making it available as a distinct NUMA node and re-designing an application
can improve the performance. Based on the MCDRAM utilization, memory
bandwidth bound applications are of two types:
\begin{itemize}
    \item Applications where the entire memory can fit in the MCDRAM; and
    \item Applications that are capable of partitioning memory usage into
    bandwidth critical and normal part, with the bandiwdth critical part
    allocated on MCDRAM.
\end{itemize}
The current OpenSHMEM memory model doesn't handle both the above mentioned
categories for utilizing MCDRAM in flat mode configuration. And it is not
for OpenSHMEM to handle anything special for cache mode configuration.

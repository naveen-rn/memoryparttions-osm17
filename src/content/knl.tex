\section{Intel KNL Architecture}
\label{src:knl}

Intel's popular Many Integrated Core (MIC) architectures are marked
under the name Xeon Phi and the second generation processors code
named Knights Landing (KNL) are used as an example for emerging
architectures with tiered memory systems in this paper. This section
provides a brief introduction to the KNL processor architecture.

The Intel KNL offers at least 64 compute cores per chip with four
threads per core. Apart from offering more than 2 TF double precision
floating point performance, each KNL core supports two 512-bit vector
units and AVX-512 SIMD intructions.

\subsection{Tiered Memory System in Intel KNL architecture}
\label{src:knl/mmodel}
KNL offers an on-package high bandwidth memory technology
called Multi-Channel DRAM (MCDRAM) in addition to traditional DRAM.
MCDRAM offers high bandwidth up to 4X more than DDR4, but with
limited capacity (up to 16GB) when compared to DDR4 (up to 384GB).
MCDRAM can be configured as a direct mapped L3 cache layer or as a
distinct NUMA node. Configuring the MCDRAM as an L3 cache layer is a
convenient way to port existing applications on to KNL based systems.
However, configuring the MCDRAM as a distinct NUMA node and re-designing
an application to fully take advantage of the high bandwidth offered
by the MCDRAM can improve the performance of memory bandwidth bound
applications.

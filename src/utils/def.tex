% add some basic definitions
\hyphenation{op-tical net-works semi-conduc-tor OpenSHMEM}

\DeclarePairedDelimiter\ceil{\lceil}{\rceil}
\DeclarePairedDelimiter\floor{\lfloor}{\rfloor}

\renewcommand{\algorithmcfname}{ALGORITHM}
\SetAlFnt{\small}
\SetAlCapFnt{\small}
\SetAlCapNameFnt{\small}
\SetAlCapHSkip{0pt}
\IncMargin{-\parindent}

\newcommand{\ie}{\textit{i}.\textit{e}.}
\newcommand*\samethanks[1][\value{footnote}]{\footnotemark[#1]}

\lstset{language=C}

\newcommand{\keywords}[1]{\par\addvspace\baselineskip
\noindent\keywordname\enspace\ignorespaces#1}

\makeatletter
\renewcommand\section{\@startsection{section}{1}{\z@}
                        {-8\p@ \@plus -4\p@ \@minus -4\p@}
                        {6\p@ \@plus 4\p@ \@minus 4\p@}
                        {\normalfont\large\bfseries\boldmath
                        \rightskip=\z@ \@plus 8em\pretolerance=10000 }}
\renewcommand\subsection{\@startsection{subsection}{2}{\z@}
                        {-8\p@ \@plus -4\p@ \@minus -4\p@}
                        {6\p@ \@plus 4\p@ \@minus 4\p@}
                        {\normalfont\normalsize\bfseries\boldmath
                        \rightskip=\z@ \@plus 8em\pretolerance=10000 }}
\renewcommand\subsubsection{\@startsection{subsubsection}{3}{\z@}
                        {-4\p@ \@plus -4\p@ \@minus -4\p@}
                        {-1.5em \@plus -0.22em \@minus -0.1em}
                        {\normalfont\normalsize\bfseries\boldmath}}
\makeatother

\makeindex

% specific definitons
\newcommand{\csma}{Cray SHMEM}
\newcommand{\osma}{OpenSHMEM}
\newcommand{\sma}{\emph{SHMEM }}
\newcommand{\pe}{\emph{PE }}
\newcommand{\pes}{\emph{PEs }}
\newcommand{\activeset}{\emph{Active Set }}